\section{Estat de l'art sobre les tecnologies utilitzades}
{
    L'interés en les transmissions en directe ha fet que molts esforços d'empreses i grups de recerca
    es focalitzin en desenvolupar noves tecnologies per fer front a la creixent demanda en termes de qualitat
    i al mateix temps que donar servei cada cop a més clients. Les tecnologies que permeten aquestes transmissions
    es centren principalment en dos grups: protocols i codificacions. No obstant, en el nostre cas nomès ens centrarem
    en els protocols, perquè aquests poden funcionar independentment de les codificacions.

    Per tal d'entendre millor el projecte, en els següents apartats s'explicaran el principis i objectius de les
    tecnologies implicades. A l'annexe 1 hi ha un resum extens de l'evolució de les tecnologies implicades fins a arribar 
    la creació de QUIC o l'ús que es fa avui en dia de multicast. Es recomana donar una ullada si hi ha alguna sigla o tecnologia 
    que no s'entén.
}

\subsection{Contexte de QUIC}
{
    Transmission Control Protocol (TCP) és un protocol dissenyat per enviar un fluxe de dades entre dos punts. Les dades son enviades
    pel sistema TCP que s'encarrega d'assegurar que les dades arribaran a l'altre extrem extactament iguals o que en el cas de que no
    fos així indicar que hi ha un error a la connexió.

    Per aconseguir això, TCP parteix les dades en petits paquets i afegeix una sèrie de capçaleres. Aquestes dades extres inclouen un 
    número de seqüència que s'utilitza per detectar quan un paquet s'ha perdut o arribar tard (Out-of-order) i una suma de comprobació
    s'utilitza per detectar que hi ha hagut un error a les dades. Quan algún problema ocorreix, TCP automàticament utilitza un algorisme
    que es diu Automatic Repeat Request (ARQ) en el que es notifica al emisor que reenvïi el paquet perdut o mal transmés.

    En la majoria d'implementacions, quan hi ha TCP detecta un error pararà la transmissió fins que l'error s'hagi resolt o consideri
    que la connexió ha fallat. En cas d'utilitzar un connexió multiplexada en diversos fluxos de dades, com en el cas de HTTP/2, tots
    es veuran afectats i es pararan encara que nomès hi hagi hagut un error en un fluxe. Per exemple, si hi ha un error carregant una
    imatge d'una pàgina web, tota la resta de fluxos hauran d'esperar fins que el problema s'hagi solucionat. Això es coneix com el 
    problema de encapçalament de linea o head-of-line blocking.

    TCP està dissenyat com una tuberia de dades en la qual no entén el que s'està enviant. Si es requereixen necessitats adiccionals 
    com encriptació (TLS), aquests s'han de fer des de capes superiors a TCP, utilitzant TCP per comunicar-se amb l'altre extrem 
    que utilitza software semblant. Cadascún d'aquests protocols necessita el seu propi \textit{handshake}. Normalment això implica
    que hi hagin varis intercanvis de paquets de sol·licitud i resposta per establir la connexió fins que aquesta finalment s'ha
    establert. Això pot arribar a ser un molt greu problema en sistemes de comunicacions amb alta latència, ja que pot significar
    sobrecàrrega de dades de control en el transcurs de la connexió.
}

\subsection{QUIC}
{   
    QUIC és un nou protocol d'Internet que es diu que canviarà com funciona actualment Internet. Daniel Stenberg, creador de
    l'aplicació curl i un dels contribuidors a l'estàndard de QUIC, va dir que fins ara HTTP s'ha fet sobre TCP i això ha
    implicat grans limitacions, però amb la vinguda d'aquest nou protocol (QUIC), les regles del joc podràn canviar 
    millorant les velocitats i demés aspectes de l'Internet actual. 
    
    A principis de la dècada passada, a Google varen començar a platejar un nou estàndard pensat per substituir TCP i al mateix
    temps simplificar i millorar coses com la seguretat o que l'establiment del canal fos el més ràpid possible inclús arribant
    al punt de que en la primera resposta el servidor pogués donar informació útil al client. Amb aquesta idea en ment varen
    començar a treballar en el que anomenaren Quick UDP Internet Protocol o QUIC. 

    Avui en dia es tracta d'un estàndard redactat publicat en els RFC 8999 (Version-independent Properties of QUIC),
    \textbf{RFC 9000} (QUIC: A UDP-Based multiplexed and Secure Transport), RFC 9001 (Using TLS to Secure QUIC) i RFC 9002 (QUIC Loss
    Detection and Congestion Control). L'estàndard de QUIC de Google es coneix com gQUIC a dia d'avui i està pràcticament deixat.
    Es recomana l'ús de l'estàndard proposat per l'IETF que va sortir en Maig de 2021.

    A l'inici aquest protocol estava pensant per treballar pràcticament en la capa de transport (capa L4 del model OSI). No obstant,
    varen trobar un greu problema amb com s'havien implementat els dos protocols predominants fins el moment, TCP i UDP: estaven 
    implementats en la majoria de nuclis dels sistemes operatius. Això implica que el desplegament o actualització d'un protocol 
    sigui extremadament lento i pugui tardar més de una dècada. Sobretot també s'ha de pensar que hi ha molts firewalls que 
    no accepten tràfic que no sigui d'aquests dos protocols.

    Al veure, aquesta situació varen optar per una solució en una capa superior, la qual normalment està implementada en l'espai d'usuari i 
    sol ser més fàcil d'actualitzar a priori\footnote{En teoria hauria de ser així, però per exemple en el cas de TLS 1.2 va ser un problema
    molt gran la seva actualització. Recomanat el article de Cloudflare al respecte. Està a la bibliografia}. Donat que UDP és un protocol
    tan sencill i que delega pràcticament totes les tasques de retransmissions o control de fluxe a capes superiors, varen decidir fer ús
    d'aquest per assegurar la interoperabilitat en els nodes intermitjos que no entenien aquest nou protocol. Aconseguit saltar aquesta 
    gran barrera, ja varen poder començar a treballar en els aspectes d'aquest nou protocol.
}

\subsubsection{QUIC: establiment de la connexió}
{
    Un dels aspectes relevant de QUIC és que utilitza menys paquets per arribar a establir la connexió inicialment. La filosofia per poder
    arribar a aconseguir això és que es realitzi el \textit{handshake} tant a nivell de connexió (imitant TLS) com a nivell de seguretat (TLS).

    Si comparen el \textit{handshake} de QUIC, que inclou implícitament el \textit{handshake} de TLS 1.3, amb el \textit{handshake} de TCP + TLS 1.2
    podrem observar que el número de paquets intercanviats entre els dos extrems és molt major en el segon cas. El problema resideix en que al mirar 
    als dos protocols com dos protocols diferents, llavors l'establiment de la connexió primer s'ha de fer per TCP i després per TLS. A QUIC es 
    proposa fer els dos al mateix temps.

    \begin{figure}%
        \centering
        \subfloat[\centering label 1]{{\includegraphics[width=5cm]{img1} }}%
        \qquad
        \subfloat[\centering label 2]{{\includegraphics[width=5cm]{img2} }}%
        \caption{2 Figures side by side}%
        \label{fig:example}%
    \end{figure}





    És més, en cas de que
    s'hagi establert prèviament la connexió i es vol reanudar pot arribar al punt de rebre dades des del \ac{RTT}.  
}

\subsubsection{QUIC: Multiplexat}
{

}

\subsubsection{QUIC: Control de fluxe a nivell de connexió i de paquets de dades}
{
    
}

\subsubsection{QUIC: Multiplexat}
{
    
}

\subsubsection{QUIC: Autenticació i encriptació de la capçalera i càrrega útil}
{
    
}

\subsubsection{QUIC: Migració de la connexió}
{
    
}
