\clearpage\section{Introduction}

{
    The main purpose of this thesis is to test QUIC protocol over Multicast in a simulated environment in order to analyse the benefits and withdraws of the current
    way of delivering live-streaming content through networks, which is using unicast and TCP. An HTTP Radio Station will be deployed using NGHQ, the only avalaible
    open-source QUIC Multicast library. It does not have a stable version, so for future implementations some specifications may be different.

    In order to replicate the internet environment, it has been developed a similar sceneario of the IP Multicast Practice from TCGI\footnote{TCGI : Transport, Control
    i Gestió d'Internet} but developed in VNX\footnote{VNX : \textbf{V}irtual \textbf{N}etwork over Linu\textbf{x}} instead of SimCTL and using an Ubuntu 18.03 LTS
    instead of Debian 5.

    A unicast simple server using QUIC was also done using the NGTCP2 library, which is stable. It will be used as the reference for nowadays servers using the RFC 9000 QUIC.
    
    Due to the complexity of the technologies involved and the differences of the approach using multicast instead of unicast it was needed to read a lot of documentation
    such as RFCs. During that time, it has been done the initial tests with the NGHQ library with some examples which are included with it (a simple sender-receiver application).
    Unluckly, it was found a small problem analysing the packets: wireshark did not detect them as QUIC. In order to be able to analyse those packets, a dissector using Lua
    have been developed. After the creation of the dissector, the small radio-station was developed and tested.
}

\subsection{Gantt Diagram}
\label{ssec:gantt}
\begin{figure}[H]
    \centering
    %\includegraphics[width=13cm]{img/diagram_gantt.png}
    \begin{adjustbox}{max totalsize={\textwidth}{.8\textheight},center}
    \begin{ganttchart}[
        hgrid,
        vgrid={*{6}{draw=none},{dotted}},
        vrule/.style={very thick, red},
        x unit=0.125cm,
        time slot format=isodate,
        time slot unit=day,
        calendar week text = {W\currentweek{}},
        bar height = 0.6, %necessary to make it fit the height
        bar top shift = 0.2, %to move it inside the grid space ;)
        bar label node/.append style={align=left,text width={width("Desenvolupament ")}},
        bar incomplete/.append style={fill=cyan},
        progress label text = \relax
        ]{2021-09-13}{2022-06-05}
        \gantttitlecalendar{year, month=name, week} \\
        \ganttbar[bar/.append style={fill=cyan}]{RFCs}{2021-09-15}{2022-05-01}\\
        \ganttbar[bar/.append style={fill=blue}]{Draft}{2021-09-15}{2021-10-01}\\
        \ganttbar[bar/.append style={fill=cyan}]{multicast \& \\ QUIC}{2021-09-15}{2022-05-01}\\
        \ganttbar[bar/.append style={fill=blue}]{Documentació \\ programes}{2021-09-15}{2021-12-31}\\
        \ganttbar[bar/.append style={fill=cyan}]{Entorn VNX}{2021-09-15}{2021-12-31}\\
        \ganttbar[bar/.append style={fill=blue}]{LXCs}{2021-10-15}{2021-12-31}\\
        \ganttbar[bar/.append style={fill=cyan}]{Provisionament}{2021-10-15}{2021-12-31}\\
        \ganttbar[bar/.append style={fill=blue}]{Pràctica}{2021-10-15}{2021-12-31}\\
        \ganttbar[bar/.append style={fill=cyan}]{Documentació \\ NGTCP2}{2022-02-15}{2022-05-15}\\
        \ganttbar[bar/.append style={fill=blue}]{Desenvolupament \\ NGTCP2}{2022-02-15}{2022-05-15}\\
        \ganttbar[bar/.append style={fill=cyan}]{Documentació \\ NGHQ}{2022-02-15}{2022-05-15}\\
        \ganttbar[bar/.append style={fill=blue}]{Desenvolupament \\ NGHQ}{2022-02-15}{2022-05-15}\\
        \ganttbar[bar/.append style={fill=cyan}]{Comparativa \\ desenvolupament}{2022-02-15}{2022-05-15}\\
        \ganttbar[bar/.append style={fill=blue}]{Comparativa \\ tràfic}{2022-04-10}{2022-04-20}\\
        \ganttbar[bar/.append style={fill=cyan}]{Proposició}{2021-09-15}{2021-09-20}\\
        \ganttbar[bar/.append style={fill=blue}]{Revisió}{2021-11-10}{2021-11-15}\\
        \ganttbar[bar/.append style={fill=cyan}]{Informe}{2022-04-05}{2022-05-15}\\
        \ganttbar[bar/.append style={fill=blue}]{Presentació}{2022-05-10}{2022-05-30}\\
        %\ganttvrule{2022-05-15}{2022-05-15}
    \end{ganttchart}
\end{adjustbox}
    \caption[Project's Gantt diagram]{\footnotesize{Gantt diagram of the project}}
    \label{fig:gantt}
\end{figure}

