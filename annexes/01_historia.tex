\subsection{Internet a finals dels 90 i principis dels 2000}
{
    A l'any 1981, Jon Postel va publicar el \ac{RFC} 791 que va revolucionar les xarxes com avui en dia les coneixem.
    Aquest RFC especificava el protocol IP que permitia comunicar dues xarxes amb tecnologies diferents. Aquesta tecnologia
    és sobre la qual es basa pràcticament el que avui coneixem com Internet.
    
    No obstant, a causa de la poca fiabilitat de les linies de transmisió (la majoria eren \ac{ADSL} fins a principis dels 2000) es van
    desenvolupar 2 protocols més que avui en dia son utilitzats de manera extensiva: el protocol \ac{ICMP}, RFC 792, i
    el protocol TCP, RFC 793. ICMP es va desenvolupar per poder fer un diagnòstic automàtic per saber l'estat de la xarxa/connexió i \ac{TCP}
    es va desenvolupar per guanyar fiabilitat, ja que afegia conceptes com la retransmissió d'un paquet o el \ac{ACK} per confirmar que 
    un paquet havia arribat correctament.
}

\subsection{La creació de multicast i el seu ús actual}
{
    Durant molts anys i encara avui en dia, la majoria de comunicacions que es fan via Internet son unicast. De fet, el protocol IP 
    inicialment estava desenvolupat per ser unicast. No obstant, l'any 1986 un estudiant de doctorat de la universitat de Stanford 
    anomenat Steve Deering va proposar amb la publicació del RFC 988 una extensió d'IP perquè les comunicacions poguessin ser de un 
    per a molts. És dir, que a través d'una única direcció es pogués distribuir el mateix contingut a tots els receptors que ho
    volguessin. La idea es podria dir que està basada en com funcionen actualment la radio o la televisió en la qual una antena emet
    la senyal i tothom que la vulgui rebre pot instal·lar una antena receptora per rebre la emissió.

    Encara que a priori sembla una tecnologia molt sencilla, en realitat en xarxes com les d'Internet pot arribar a ser força complicat.
    Entre d'altres problemes aquesta tecnologia és pot fiable en termes de comunicar que el paquet ha arribat correctament al emissor, ja
    que el emissor emet pero "no espera" que ningú li respongui. Normalment s'utilitzen canals alternatius que fan ús de unicast per aquest
    tipus de casos.
    
    Un altre problema molt gran de multicast té que veure amb la congestió, ja que al no tenir un mecanisme de feedback, com sí te TCP, llavors
    no es pot gestionar de manera tan sencilla. Ademés s'ha d'afegir que enviar el mateix paquet a tothom com si només fos un client,
    però enviar-lo per diferents xarxes amb diferents situacions sense saber si arribarà bé o no encara complica més el tema . Hi ha protocols
    com Scream que sí que permeten aquesta gestió de la congestió i demés, però encara així tampoc és gens sencill.

    Com a últim gran problema d'aquest és que pot generar bucles o també dit "tempestes multicast", en les quals el paquet s'emet d'un router a un altre
    fent un bucle si aquest poden arribar a tenir una topologia d'anell. És un problema que no obstant es pot solucionar amb protocols com \ac{PIM}.

    Per poder arribar a fer ús de multicast, com es pot intuir fàcilment, no es pot fer ús de la tecnologia com TCP. Per utilitzar IP multicast hi ha
    dos tecnologies que s'han d'utilitzar avui en dia: per un part IGMP i per l'altra UDP.
}

\newpage 
\subsection{IGMP}
{
    \ac{IGMP} o Internet Group Management Protocol és un protocol que serveix per avisar al router a través del qual es pot rebre la senyal,
    o en el seu defecte a l'emisor, que es vol rebre la emissió. És un protocol molt sencill en el qual únicament s'avisa al node més pròxim teu que pot enviar
    la senyal quina és la direcció multicast que es vol escoltar i el port, ademés d'altres petits paràmetres. Aquest també permet que es demani que 
    la emissió sigui d'una IP d'origen concret (IGMP v3).
}

\subsection{UDP}
{
    Donada la gran popularitat de TCP, en el seu moment molts dissenyadors de sistemes operatius varen decidir implementar aquest en el nucli; el cas més
    famós és el de Linux. Per una part, aquesta decisió va fer que l'eficiència dels sistemes millorés de manera molt notòria i que molts servidors de 
    l'època poguèssin donar el servei de manera significativament millor. No obstant, aquesta decisió que a priori sembla bona per contra va implicar que també tots
    els sistemes d'enrutament, firewalls, balancejador de càrrega i demés confïin que sempre s'utilitzi aquest protocol i no deixant passar altres protocols.
    
    Per sort, uns anys abans, a l'any 1980, Jon Postel també va publicar el RFC 768 en el qual s'especifica un altre protocol alternatiu a TCP; o millor dit,
    que TCP va substituir. \ac{UDP} o User Datagram Protocol és un protocol el qual està just per damunt de IP en el model \ac{OSI}, a la capa de
    transport. Actualment son els dos protocols principals en aquesta capa i que pràcticament tots els equips d'Internet utilitzen.

    UDP és un protocol força sencill, ja que únicament ha de passar el paquet en qüestió al port de destí indicat i poc més. Es podria dir que delega totes les
    tasques de control de fluxe, retransmissió, etcètera a capes superiors. En el cas de protocols de capes superiors que volen implementar aquestes eines a la 
    seva manera o que directament no volen utilitzar-les per raons vàries, TCP pot arribar a ser un problema i UDP esdevé la solució. La gran avantaje d'aquest
    protocol és que també està implementat a pràcticament tots els nuclis de sistemes operatius del món perquè és molt sencill encara que no s'utilitzi tant.

}

\subsection{Com accedim a una pàgina web?}
{
    A l'any 1989 Tim Berners-Lee i un grup d'enginyers del \ac{CERN} varen observar que els científics tenien un problema molt particular a la hora d'accedir
    als documents: la majoria d'aquests estaven fets en paper i si volien accedir per una cosa puntual al final perdien molt de temps, ademés que l'accés 
    en n'aquests a vegades podia ser força complicat. Al veure aquest problema varen proposar una solució que avui en dia conèixem com navegador web i HTTP.

    Bàsicament, el que proposaren era una eina en la qual els científics podien escriure els seus documents de manera sencilla a l'ordinadors i compartir-lo
    amb tothom al mateix temps que es podia accedir a n'aquest de manera ràpida des de qualsevol lloc que tingués connexió amb el servidor on es deixava el 
    document. L'eina de visualització del document és el que conèixem com navegador web, que avui en dia tenen la mateixa base pero defereix molt al que eren
    antigament, i el protocol per publicar o demanar el document es diu HTTP.

    Va ser tal la popularitat d'aquestes dues eines que avui en dia es utilitzat per qualsevol persona que té accés a internet. S'utilitza
    ja sigui per mirar una pàgina web de noticies, mirar un video de cuina o inclús per fer transaccions bancàries. Aquest protocol tant extensament utilitzat 
    va pràcticament en texte pla i viatja per internet sense cap tipus de encriptació per si sol.  De fet, fins al principis del 2000 s'utilitzava pràcticament
    així. No obstant, amb el creixement de pàgines web de comerç elèctronic es va veure que transferir dades bancàries en pla a través de la xarxa tal vegada no 
    és la millor idea. Per intentar solucionar aquest problema varen sorgir \ac{SSL} o Socket Security Layer o el seu predesor que avui en utilitzem: TLS o Transport
    Security Layer.
}

\subsection{I la seguretat es tornà una necessitat}
{
    TLS o Transport Security Layer és un protocol de seguretat que s'ubica en la capa de sessió o capa L5 en el model OSI i que té la funció d'encriptar les
    comunicacions entre dos interlocutors o, dit d'un altra manera, entre el client i el servidor. Aquest protocol està basat en SSL que és el seu precursor.
    La idea bàsica de funcionament d'aquest és que entre els dos punts s'acorden unes claus d'encriptació i el tipus d'encriptació.
    
    Per fer la connexió inicialment el model és pràcticament igual que el de TCP. Per iniciar, primer el client li diu al servidor que vol establir una connexió 
    amb ell ("client hello"), després aquest li respon que ha rebut el missatge i també li contesta ("server ACK/ server hello") i finalment el client li diu al
    servidor que a rebut el seu missatge ("client ACK"). D'aquesta manera s'ha pogut establir la connexió. A part també una validacions i certificacions a part 
    per poder autenticar que el servidor és qui realment diu ser.

    A la següent imatge es pot veure en cert detall l'establiment de la connexió. En els tres primer paquets, que están en blau, es pot observar com s'estableix 
    la comunicació TCP i en els següents paquets, que están en groc, com s'estableix la comunicació TLS. És important remarcar que perquè es pugui establir la 
    comunicació en una capa superior primer s'ha d'establir en un capa inferior com és el cas que es veu amb TCP i TLS, primer s'estableix TCP i després TLS. No
    obstant, sí que es pot mantenir la sessió com ja es veurà quan es parli de QUIC.

    \begin{figure}[H]
        \label{fig:tcp_tls_handshake}
        \centering
        \includegraphics[width=15cm, height=7.2cm]{img/02_stateofart/tcp_tls_handshake.png}
        \caption[TCP i TLS handshake]{\footnotesize{Handshake de TCP en blau i handshake de TLS en groc. Imatge de cloudflare.}}
    \end{figure}

}

\subsection{Pero la seguretat també pot ser un problema}
{
    Al veure el diagrama anterior, un es pot donar compte d'un fet molt important: la comunicació és només entre dos terminals. Això vol dir que només es pot
    encriptar la comunicació entre dos nodes? La resposta és sí, pero no.

    Els algorismes actuals de criptografia estan pensats per ser utilitzats entre dos terminals únicament; és dir, la comunicació ha de ser punt a punt. D'aquesta
    manera, en cas de que no sigui un algorisme romput o que cap dels dos terminals estigui corrupte, la transferència de dades estarà encriptada i cap punt
    intermig podrà llegir l'informació. Això és molt important a Internet, ja que és una xarxa pública i la que si s'envien les dades de manera segura pot haver-hi 
    problemes greus sobretot per plataformes de comerç electrònic, bancs o inclús empresarials.

    No obstant, si tirem la mirada enrera podrem veure un cas en el que la criptografia es feia de un per a molts (multicast). Als anys 90 i principis del 2000 moltes
    empreses de televisió oferien el fútbol i altres canals pagant. La manera de transmetre la televisió en el millor cas en aquella época era via antena, terrestre i
    satèl·lit, o via cable. Donat que el cable podia arribar a ser car, generalment s'enviava via antena. El problema? Que qualsevol persona podia rebre la emisió.

    Per evitar que la gent no pagués i disfrutás del servei gratuitament, aquestes empreses varen decidir recurrir a la criptografia. En el seu cas, varen començar
    a distribuir unes tarjetes on hi havia una clau privada que desencriptava la senyal per poder disfrutar d'aquests canals. 
    
}

\subsection{Borrador}
{
    Per altra banda, s'observa que la quantitat de paquets per poder establir la connexió és molt alt, encara que sigui nomès
    per enviar un sencill "Hola mundo!" en HTML. És més, en el cas de que una persona estigui en el mòvil de casa seva mirant un directe a través d'una pàgina web
    i decidís sortir de casa continuant mirant el directe  \textbf{EXPLICAR PROBLEMA ESTABLECIMIENTO - MUCHOS PAQUETES}
}